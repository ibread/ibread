% 定义所有的eps文件在 figures 子目录下
\graphicspath{{figures/}}

\chapter{基于事务的层次化形式验证方法}
\label{cha:hierarchy}

% 需要包括的内容: 
%    背景知识简介( Model Checking, SMT, 符号模型检验(SMV) )
%    我的方法(层次化研究方法)
%    初步试验结果及分析
%    结论和将来的工作

\section{形式化验证方法简介}
\label{sec:hierarchy-intro}

一般情况下,形式验证可以分为等价性验证和性质检验。

\subsection{等价性验证}
\label{sec:equal}
% TO-DO: 介绍等价性验证

\subsection{性质检验}
\label{sec:property-checking}
% TO-DO: 介绍性质检验

\subsection{形式验证方法层次}
\label{sec:fv-categories}

% 介绍下形式验证的层次, Gate-Level => RTL => Transaction Level
根据``摩尔定律'',集成电路(Intergrated Chips)上可容纳的晶体管数目,约每隔18个月便会增加一倍,性能也
将提升一倍。因此随着集成电路技术的不断发展,片上系统(System-on-Chips)的复杂度和规模都随之越来越大。
相应的,对相关电子设计的验证工作也面临着很大的挑战。这种情况下,传统的集中在寄存器传输级(RTL)的验证
方法就在验证效率方面显得捉襟见肘:一方面,将大规模的设计在相对底层的寄存器传输级验证需要大量的时间和
精力;另一方面,寄存器传输级的电子设计一般在后期才能够成形以供验证,这就势必提高了修正设计失误的代
价,从而影响了上市时间,严重的甚至会导致整条产品线上市的严重滞后,从而造成巨大的经济损失。因此,业界
目前普遍比较关注将验证层次提高

\section{层次化验证方法}
\label{sec:hierarchy-details}
为了比较清晰地介绍层次化验证方法,我们将在下文中结合具体的验证实例来进行阐释。

\subsection{待验证设计}
\label{sec:hierarchy-duv}

我们选用的待验证设计(DUV, Design under Verification) 是DTL(Device Transaction Level)协议。DTL协议是
由Philips提供的一种用于点对点的同步数据传输协议 \cite{DATE09_Moll},支持单字节传输和块数据传输,以及
错误信号发射以及子字(subword)操作。除此之外,DTL协议还为特定的应用提供了扩展接
口和信号,比如:

\begin{itemize}
  \item 寻址模式: 为块数据传输所提供的包装、固定以及递减地址。
  \item 二维块操作: 用于操作内存中以二维形式存储的数据。
  \item 安全操作: 标明某个事务是否安全。
  \item 缓冲区管理: 允许组件请求写缓冲区,或者在数据到达目的地后发送通知。
\end{itemize}

基本的DTL协议由24个信号组成,其中6个用于扩展特性,比如说缓冲区管理或者二维数据操
作\cite{Book_Pasricha}。为了描述的简洁起见,我们对原始的基本DTL协议进行了简化。在简化后的协议中,仅保
留字节传输支持,并且不支持任何扩展。这样一来,24个信号中仅有12个需要保留。读/写操作是总线协议中最为重
要的两种操作,因此所保留的12个信号中大多与这两种操作相关,这一点在后文中会进一步详细阐述。

除了简化的工作以外,为了更好地描述层次化验证的思想,我们在原始的DTL协议之上又增加了一层扩展。除了基
本的读写操作之外,我们引入了并行的读/写操作,并为此设计了互斥和同步机制。其具体内容如下:

\begin{extension} \label{extension:multi-read}
当且仅当没有写操作访问同一地址时,多个读操作可以同时进行。
\end{extension}
\begin{extension} \label{extension:multi-write}
当且仅当没有读/写操作访问同一地址时,多个写操作可以同时进行。
\end{extension}


为了支持如上两种扩展,两个信号被添加到原始的信号集中: {\em is\_trans} 标示当前是否有正在进行的事
务;{\em is\_rd} 标示当前正在进行的事务是写操作还是读操作。经过简化和扩展之后,共有14个信号被引入,
并根据其用途划分成四组,具体请参见表~\ref{tab:dtl}。在后文的描述中,我们将改造后的DTL协议称为DTL-S协
议。

\begin{table}[ht]
  \caption{DTL-S协议信号和功能组}
  \label{tab:dtl}
  \centering
  \begin{tabular}{|c|c|}
  \hline
  \hline
  功能组  & 信号\\ [0.5ex]
  \hline
  命令    & cmd\_valid, cmd\_accept, cmd\_finish\\
  \hline
  写操作  & wr\_valid, wr\_addr, wr\_data, wr\_accept\\
  \hline
  读操作  & rd\_valid, rd\_addr, rd\_data, rd\_accept\\
  \hline
  互斥操作  & is\_trans, is\_rd\\
  \hline
  
  \hline\hline
 \end{tabular}
\end{table}


根据信号的名称,我们可以很自然地理解其功能。比如{\em cmd\_valid}信号用来指示命令是否生效,而{\em
  wr\_addr} 则表示写操作的地址。DTL-S协议的整体工作流程如下: 首先发送命令并在验证后设置其为有效;然
后再根据命令的具体内容进行读/写操作。

\subsection{待验证性质}
\label{sec:hierarchy-puv}
在验证过程中需要关注的性质大致分为两类: 安全性和活性。在介绍这两种待验证性质之前,我们有必要介绍一下
性质的表示方法---时态逻辑(Temporal Logic)。

\subsubsection{性质表示: 时态逻辑}
\label{sec:temporal-logic}
为了形式化地表示待验证的性质,时态逻辑的概念被引入,它是一个形式化的表达式,可以用来表示系统目前所处状态的
性质以及状态迁移序列的性质。按照时间展开的不同结构,时态逻辑可以进一步划分为线性时态逻辑(Linear
Temporal Logic)和分支时态逻辑(Branching Time Logic)。这两者的区别类似于确定性状态机和非确定性状态机
之间的区别: 对于线性时态逻辑(LTL),每个时刻的未来只有一个时刻可以到达;而对于分支时态逻辑(BTL)来说,顾名
思义,每个时刻的未来都有多个时刻可以到达。下面我们分别对其进行介绍。

\paragraph{线性时态逻辑LTL}
\label{sec:ltl}

线性时态逻辑的组成部分包括命题逻辑和时态操作符,其中时态操作符包括四种\cite{Book_Bianjn}:

\begin{itemize}
\item $\mathbf X$ ``次态(next time)''操作符
\item $\mathbf F$ ``最终(eventually)''操作符
\item $\mathbf G$ ``全程(globally)''操作符
\item $\mathbf U$ ``直到(until)''操作符
\end{itemize}

在这四种操作符中,$\mathbf X$,$\mathbf F$,$\mathbf G$为一元操作符,而$\mathbf U$为二元操作符。下面
我们结合具体的例子来说明这四种时态操作符的作用。假设有两种命题$req$和$ack$,其中$req$表示请求,而
$ack$则表示对$req$请求的响应。

\begin{itemize}
\item $\mathbf{X}\ ack$ 请求将会在下一个时刻被响应
\item $\mathbf{F}\ ack$ 请求最终会被响应
\item $\mathbf{G}\ req$ 请求将会一直被保持
\item $p\ \mathbf{U}\ q$ 在响应之前,请求将会一直被保持
\end{itemize}

\paragraph{计算树逻辑CTL}
\label{sec:ctl}

在分支时态逻辑(BTL)中,最常用的是计算树逻辑(Compunational Tree Logic),它所关注的是在状态迁移中具有
分支路径的时态逻辑。为了支持这个特性,其表示方式相比线性时态逻辑LTL,增加了两个路径量词:

\begin{itemize}
\item $\mathbf A$ 全称量词,用于表示状态迁移中的所有的路径
\item $\mathbf E$ 存在量词,表示所有可选路径中至少存在一条路径
\end{itemize} 

因此,通过结合两种路径量词和四种时态操作符,我们在使用计算树逻辑CTL时,共有8种时态操作符可以使用,分
别是$\mathbf AX$,$\mathbf EX$,$\mathbf AF$,$\mathbf EF$,$\mathbf AG$,$\mathbf EG$,$\mathbf
AU$ 和 $\mathbf EU$。

% TO-DO: 这里还可以对8种时态操作符进行解释

\subsubsection{安全性}
\label{sec:safety}

安全性(Safety)指的是某危险事件永不发生,或者某事件永远满足。比如说,安全性可以用来保证系统的互斥和同
步机制的正确实现,比如说资源的互斥使用。另外也可以用来保证一些特定的性质,比如说请求在响应前必须保持。

用线性时态逻辑LTL举例说明的话, $\mathcal{G} \neg(ack1 \wedge ack2)$表示响应$ack1$和$ack2$是互斥的,不可
能同时发生。$\mathbf{G}\ (req \rightarrow ( req\ \mathbf{U}\ ack))$ 则表示一旦有请求$req$产生,那么
在响应$ack$出现之前,请求$req$必须一直保持。

在本例中,我们需要保证的是DTL-S协议中互斥机制的正确实现,也就是说,扩展~\ref{extension:multi-read}
和扩展~\ref{extension:multi-write} 中所描述的性质必须被满足。比如说,假设在具体的系统实现中,有两个
处理器$p1$和$p2$可以通过总线访问同一块内存,其中采用的总线协议是DTL-S,具体请参加图~\ref{fig:shared_memory}。

\begin{figure}[hl]
\centering
\includegraphics[scale=0.6]{shared_memory}
\caption{共享存储器访问模型}
\label{fig:shared_memory}
\end{figure} 

为了保证互斥操作的正确实现,即不会有多个读写或多个写操作同时访问同一内存地址,我们需要验证DTL-S协议能
够满足公式~\eqref{equ:safety}所示的性质。也就是说,在$p1$和$p2$访存地址相同的情况下,两者只能同时是读操作,否
则根据互斥机制,两个操作不能同时被接受。

\begin{equation} \label{equ:safety}
\begin{split}
\mathbf{G}!&((p1.wr\_addr == p2.wr\_addr\ \&\ p1.wr\_accept\ \&\ p2.wr\_accept)\\
&|\ (p1.wr\_addr == p2.rd\_addr\ \& \ p1.wr\_accept\ \&\ p2.wr\_accept)\\
&|\ (p2.wr\_addr == p1.rd\_addr\ \& \ p2.wr\_accept\ \&\ p1.wr\_accept))
\end{split}
\end{equation}

\subsubsection{活性}
\label{sec:liveness}

活性(Liveness)所关注的是某事件最终必须要发生。比如说对于实际应用中的大部分系统来说,我们希望如果用户
给出请求以后,系统最终能够给与回应,用线性时态逻辑LTL表示的话,就是$\mathbf{G}(req \rightarrow ack)$。

在本例中,我们也需要关注相似的性质,也就是说,DTL-S协议应当能够满足公式~(\ref{equ:liveness})所示的性
质。也就是说,只要有命令被确认为有效,就一定会继而触发读/写命令。

\begin{equation} \label{equ:liveness}
\begin{split}
\mathbf{G}(
&(p1.cmd\_accept \rightarrow p1.wr\_valid\ |\ p1.rd\_valid)\\
&(p2.cmd\_accept \rightarrow p2.wr\_valid\ |\ p2.rd\_valid))
\end{split}
\end{equation}

注意,虽然公式~(\ref{equ:safety})和公式~(\ref{equ:liveness})所示的性质必须经过验证过程来判断是否满
足,但是我们要注意到在高层的验证过程中某些特定的信号,比如说公式~(\ref{equ:safety})中的
$p1.wr\_addr$,会在抽象过程中消失。这样的话,我们有可能需要在进一步验证性质是否成立的时候对抽象后的
模型重新进行细化。这一点我们将在~\ref{sec:hierarchy-method}中进行详细的阐述。


\subsection{层次化验证方法}
\label{sec:hierarchy-method}

寄存器传输级(RTL)的设计由于处于相对底层,不可避免的要引入大量的信号,这在目前日益复杂的电子设计中表现
地尤为明显。这样一来,对寄存器传输级设计的验证问题规模就日益扩大,从而给验证技术带来了不小的难题。

一方面,由于电子设计的规模日趋庞大,基于寄存器传输级的设计的验证工作要耗费大量的时间和精力,这正是基
于事务的设计/验证目前受到广泛关注的原因。事务级(Transaction Level)相对于寄存器传输级(RTL)来说抽象层次
更高,因此待验证模型的规模会大大减少,验证效率则会相应地有很大提高。与此同时也可以将验证工作提前到设
计的相对初期来做,可以更早地发现设计失误并予以解决,从而可以尽可能地避免由于设计失误造成的损失,同时
大大缩短上市时间。

另一方面,在待验证的诸多性质中,如果有一些细粒度的性质,需要设计到具体的信号,显然就不适合在高层做验
证。因此事务级建模TLM以及基于事务的验证方法就不适用于这一类性质,验证的完整性也无法保证。在验证这类
涉及到具体信号的性质时,比较理想的方法则是将待验证设计在寄存器传输级(RTL)建模,然后再进行验证。

综上所述,单一地在寄存器传输级(RTL)和事务级(TL)做验证都各有利弊: 寄存器传输级的验证可以更好地保证验证
的完整性,但验证效率会大打折扣;事务级的验证则恰恰相反。而如果可以将两者结合起来,使其可以互相补充,
则可以在很大程度上同时保证验证的完整性和效率。这正是我们提出层次化验证方法的动机。

简单来说,我们会将待验证性质根据粒度划分为两部分: 粗粒度的性质比较适合于基于事务级的验证,而细粒度的
性质则比较适合于寄存器传输级的验证。在验证之初,我们优先保证验证效率,将原始设计做事务级建模,相应
的,也把待验证性质做响应的形式化表示,然后尽可能多地对所有性质做验证。但正如之前讲到的,很多性质在高
层次无法得到确切的验证结论。比如说对于保证多个进程不会同时写同一地址的安全性$P$来说,如果在寄存器传
输级的模型中我们已经可以获取该性质的反例,那么

比如对于安全性来说,验证阶段要做的工作就是检测是否会有违反该性质的反例产
生。但由于很多细节会在对原始设计的高层次建模中被抛弃,

模型的抽象过程实际上就是状态机中小状态集合成为大状态的过程。

细粒度模型的反例 一定是 粗粒度模型的反例。如果在小状态都有冲突,那么集成为大状态也会有冲突。比如说p1
和p2都写地址0x100,抽象后我们看到的行为可能会是p1和p2都处于写操作阶段。这样的话我们就无法确定是否真
的会有冲突,需要进一步细化后才可以得出结论。

粗粒度模型的反例 不一定是 细粒度模型的反例

粗粒度模型没有反例 => 细粒度模型也没有反例


\subsection{模型检验}
\label{sec:model-checking}

模型检验(Model Checking)在形式验证中占据非常重要的位置,也是近几十年以来形式验证最主要的发展力量。它
是由美国的Edmund M. Clarke 和 Allen Emerson\cite{WLP81_Clarke},以及法国的Jean-Pierre
Queille 和 Joseph Sifakis\cite{EWATPN81_Queille}分别独立提出的。模型检验发展至今,不仅在软/硬件设计的
验证领域中占有极为重要的位置,此外在通信协议、安全算法的设计方面也发挥了很重要的作用。模型检验技术的
发明者Edmund M. Clarke,Allen Emerson和Joseph Sifakis三人也因为``在将模型检查发展为被软/硬件业中广泛
采用的高效验证技术上的贡献''被授予2007年度的图灵奖。

模型检验的主体思想是,首先将要验证的对象(比如说系统或者电子设计)抽象为有限状态机(Finite Sates
Machine),待验证的性质则用时态逻辑(Temporal Logic)进行形式化的描述。由于待验证系统已经被抽象为有限状
态机,因此直观的想法就是如果我们遍历状态机中的每个状态,同时查看性质$P$在该状态是否成立,就可以判断在
整个状态空间中是否存在违反该性质$P$的反例,从而得出结论。因此模型检验的过程一般是三段式的,包括系统
建模、性质形式化表示以及验证。

模型检验相对于定理证明来说,由于实现方式和理论基础都相对要简单很多,给验证工作的进行带来了很大的便
利,也因此引领了形式验证技术的发展。但模型检验技术也并不是形式验证领域的``银弹'',它同样有自己的瓶颈。
正如上文提到的,模型检验虽然可以实现验证工作的自动化完成,实际上是通过遍历有限状态机的状态空间来判断
性质是否满足的,这样对于规模大的问题就不可避免地遇到状态空间爆炸的问题。实际上这也正是模型检验技术从
开创至今一直努力克服的问题,模型检验发展的历史,实际上就是在克服空间爆炸问题的道路上不断摸索前进的过
程。

模型检验技术发展至今,按照时间先后分别出现了显式模型检验,符号模型检验(Symbolic Model Checking)和
定界模型检验(Bounded Model Checking)三种。

\paragraph{显式模型检验}
显式模型检验就是上文中提到的显式遍历状态机状态空间的方法。

\paragraph{符号模型检验}
符号模型检验(SMV)是由K.L. Mcmillan将Randal E. Bryant提出的的有序二叉判别图(Ordered Binary
Decision Diagram,OBDD)应用于模型检验技术中,提出的一种新型的检验方法。由于OBDD可以表示状态组而不是
单个状态之间的转换关系,状态空间爆炸的问题得到很大的缓解,模型检验能够处理的问题规模有了很大的提高。
也是从符号模型检验SMV开始,模型检验技术开始走出学院,广泛应用于工业界,并获得了巨大的成功。

\paragraph{定界模型检验}
对于二叉判别图BDD来说,由于要进行状态穷举遍历,每个变量的两种取值可能(0和1)都要表现出来,因此未经化
简的BDD是一个完全二叉树,其规模与变量数呈指数关系。虽然通过多种手段对其进行化简,但存储和操作BDD仍然
需要很大的空间。因此,在定界模型检验SMV之后,Armin Biere于1999年提出了定界模型检验
(BMV)\cite{Biere03boundedmodel}。定界模型检验是基于可满足性(Satisfiability, SAT)问题提出的一种模型检
验方法。它的基本思想是将模型检验问题转化为一个可满足性问题,并借助于SAT求解器来得出结果。具体来讲,
定界模型检验会将待验证性质$P$取反得到$\neg P$,并给定一个界限$k$,将有限状态机$M$展开$k$个时序,并和待验证性质
$P$结合构成一个可满足性(SAT)问题。如果求解得知该问题满足,即原始设计在$k$个时序内满足性质$\neg P$,就说明
我们已经找到了性质$P$的一个反例。否则的话,我们需要继续提高$k$,或者在到达预定的最大上限后认为反例无
法找到。

相对于采用BDD的符号模型检验来说,由于不需要保存BDD,基于SAT问题的定界模型检验最大的优势是不存在空间爆
炸问题。而且虽然SAT问题是NP-Complete问题,但是在实际应用中仍然非常有效。而且得益于近几年SAT求解技术
的蓬勃发展,定界模型检验的效率以及可以处理的问题规模也有很大的提高。

\section{结果与分析}
\label{sec:hierarchy-results}


%%% Local Variables: 
%%% mode: latex
%%% TeX-master: "../main"
%%% End: 
